\documentclass[main.tex]{subfiles}

\begin{document}

\section{Known Issues and Limitations}\label{sec:known-issues-and-limitations}

During the analysis and development of the DApp, we identified two main limitations closely related to the user experience. 
Currently, the fact of not having common browsers that can support the opening of such domains registered on the Ethereum blockchain increases the complexity of using the platform making it to the hands of only those more experienced users.
Thus, although access to resources saved on TOR or IPFS is simplified, which are usually pointed by hash addresses is still difficult for what is explained above to have an easy mass adoption for all those users less accustomed to blockchain technology.

Secondly, there is a dependence on using digital wallets like Metamask, necessary to authenticate the user within the DApp but binding in performing even the simplest operations such as data consultation as they are deposited within the Ethereum blockchain. A possible compromise would be to save some information outside of it in order not to be tied to the use of wallets but you would lose the main purpose of the DApp.


\end{document}
